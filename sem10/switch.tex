% !TEX encoding   = UTF8
% !TEX spellcheck = ru_RU
% !TEX root = ../seminars.tex

%%=======================================
\chapter{Ветвления и циклы (продолжение)}
%%=======================================

%%=========================
\section{Цикл \texttt{for}}\label{sect:for}
%%=========================
Общая форма инструкции \code{for} в~\lang{C} задаётся шаблоном:

\begin{ccode*}{linenos=false}
for (|\mystyle{init-expr}|; |\mystyle{test-expr}|; |\mystyle{update-expr}|)
  |\mystyle{body-statement}|
\end{ccode*}

\noindent Стандарт языка~\lang{C} утверждает (за~одним исключением, освещаемом в~упражнении~\ref{ex:forcontinue}), что поведение такого цикла идентично следующему коду, использующему цикл \code{while}:

\begin{ccode*}{linenos=false}
|\mystyle{init-expr}|;
while (|\mystyle{test-expr}|)
{
  |\mystyle{body-statement}|
  |\mystyle{update-expr}|;
}
\end{ccode*}

Код, генерируемый \GCC{} для~цикла \code{for}, следует одной из~двух стратегий трансляции, рассмотренных для~цикла \code{while}, в~зависимости от~уровня оптимизации:

\begin{flushleft}\begin{minipage}[t]{0.3\textwidth}
\begin{ccode*}{label=\textrm{переход-в-середину}, frame=topline, framesep=3\fboxsep, linenos=false}
  |\mystyle{init-expr}|;
  goto test;
loop:
  |\mystyle{body-statement}|
  |\mystyle{update-expr}|;
test:
  t = |\mystyle{test-expr}|;
  if (t)
    goto loop;
\end{ccode*}
\end{minipage}\qquad\begin{minipage}[t]{0.3\textwidth}
\begin{ccode*}{label=\textrm{защищённый-\code{do}}, frame=topline, framesep=3\fboxsep, linenos=false}
  |\mystyle{init-expr}|;
  t = |\mystyle{test-expr}|;
  if (!t)
    goto done;
loop:
  |\mystyle{body-statement}|
  |\mystyle{update-expr}|;
  t = |\mystyle{test-expr}|;
  if (t)
    goto loop;
done:
\end{ccode*}
\end{minipage}\end{flushleft}



%%=====================
\paragraph{Упражнение.}
%%=====================
Функция \code{fun\_b} имеет следующую структуру:

{\newcommand*{\ans}{\ansdots}%
%
\begin{ccode*}{linenos=false}
long fun_b (unsigned long x)
{
  long val = 0;
  for (|\ans{long i = 64}|; |\ans{i != 0}|; |\ans{--i}|)
  {
    |\ans{val = (val << 1) \textbar (x \& 0x1);}|
    |\ansx{x >>= 1;}|
  }
  return val;
}
\end{ccode*}
}

\noindent \GCC{} генерирует для~этой функции ассемблерный код, показанный ниже:

\precomment/x: %rdi/
\vspace{-1.7\baselineskip}
\gasfile[firstline=5, lastline=17]{projects/sem10/fun_b.s} % asmlst -O1 fun_b.c

\noindent Разберитесь, как работает этот код, и выполните следующее:
\begin{enumIssue}
\item При~помощи ассемблерного кода восстановите пропущенные части в~\lang{C}.
%\item Объясните, почему нет ни начальной проверки условия перед выполнением тела цикла, ни прыжка к части кода, которая это делает.
\item Опишите словами, что вычисляет функция.
\end{enumIssue}



%%============================================
\section{Реализация оператора \texttt{switch}}
%%============================================
Оператор \code{switch} обеспечивает множественное ветвление на~основе целочисленного значения и особенно полезен, когда имеется большое количество вариантов. Он не~только делает код на~языке~\lang{C} более читабельным, но также допускает эффективную реализацию на~основе структуры данных, называемой \emph{таблицей переходов}. Таблица переходов~--- это массив, в~котором $i$-ый элемент хранит адрес сегмента кода, реализующего некоторое действие программы, соответствующего $i$-му варианту в~операторе \code{switch}. Для~определения целевого адреса перехода код выполняет обращение к~таблице переходов, используя индекс~$i$. Преимущество использования таблицы переходов перед длинной последовательностью иструкций \code{if-else} в~том, что время на~выполнение переключения не~зависит от~количества вариантов. \GCC{} выбирает метод трансляции оператора \code{switch}, руководствуясь количеством вариантов (\textenglish{cases}) и разреженностью значений \code{case}-меток. Таблицы переходов используются, когда имеется несколько вариантов (например, четыре и более), и они укладываются в~небольшой диапазон значений.



%%=====================
\paragraph{Упражнение.}
%%=====================
В~функции на~\lang{C}, которая показана ниже, пропущено тело оператора \code{switch}. \code{case}-метки не~образуют непрерывный диапазон, а некоторые варианты имеют несколько меток.

{\newcommand*{\ans}{\ansdots}%
%
\begin{ccode*}{linenos=false}
void switch2 (long x, long *dest)
{
  long val = 0;
  switch (x)
  {
    |\textcolor{cyan}{...}  \comm{Body of switch statement omitted}|
  }
  *dest = val;
}
\end{ccode*}
}

\noindent При~компиляции этой функции \GCC{} генерирует следующий ассемблерный код для~начальной части процедуры:

\precomment/x: %rdi/
\vspace{-1.7\baselineskip}
\begin{gascode}
switch2:
  addq  $1, %rdi
  cmpq  $8, %rdi
  ja    .L2
  jmp   *.L4(,%rdi,8)
\end{gascode}

\noindent а также код для таблицы переходов:

\begin{gascode}
.L4:
  .quad  .L9
  .quad  .L5
  .quad  .L6
  .quad  .L7
  .quad  .L2
  .quad  .L7
  .quad  .L8
  .quad  .L2
  .quad  .L5
\end{gascode}

\noindent Опираясь на~эти данные, ответьте:
\begin{enumIssue}
\item Какие значения меток были в~операторе \code{switch}?
\item Какие варианты имеют несколько меток в~\lang{C} коде?
\end{enumIssue}



%%================
\WhatToReadSection
%%================
\citeauthor[глава~3, стр.~242--250]{Bryant:2022:ru}



%%===============
\ExercisesSection
%%===============
\begin{exercise}
\item\label{ex:forcontinue} Оператор \code{continue} в~\lang{C} заставляет программу перейти в~конец текущей итерации цикла. Изложенное выше правило перевода цикла \code{for} в~цикл \code{while} требует некоторого уточнения, если в~теле цикла встречается этот оператор. Для~примера рассмотрим следующий код:

\begin{ccode}
/* Example of for-loop containing a continue statement */
/* Sum even numbers between 0 and 9 */
long sum = 0;

for (long i = 0; i < 10; ++i)
{
  if (i & 1)
    continue;

  sum += i;
}
\end{ccode}

\begin{enumIssue}
\item Что получится, если наивно применить упомянутое на~странице~\pageref{sect:for} правило перевода цикла \code{for} в~цикл \code{while}? Что будет неверным?
%
\item Как следует заменить оператор \code{continue} на~оператор \code{goto}, чтобы гарантировать, что цикл \code{while} корректно воспроизводит поведение цикла \code{for}?
\end{enumIssue}


\item Для~функции \code{switcher}, написанной на~\lang{C} и имеющей следующую структуру:

{\newcommand*{\ans}{\ansfw{7.5em}}%
%
\begin{ccode}
void switcher (long a, long b, long c, long *dest)
{
  long val;

  switch (a)
  {
  case |\ans{5}|:      /* Case A */
    c = |\ans{b \textasciicircum 15}|;
    /* Fall through */

  case |\ans{0}|:      /* Case B */
    val = |\ans{c + 112}|;
    break;

  case |\ans{2}|:      /* Case C */
  case |\ans{7}|:      /* Case D */
    val = |\ans{(b + c) << 2}|;
    break;

  case |\ans{4}|:      /* Case E */
    val = |\ans{a}|;
    break;

  default:
    val = |\ans{b}|;
  }

  *dest = val;
}
\end{ccode}
}

\GCC{}, запущенный с~ключом \code{-fno-pie}, генерирует ассемблерный код и таблицу переходов, как показано ниже:\label{ex:jumptable}

\precomment/a: %rdi, b: %rsi, c: %rdx, d: %rcx/
\vspace{-0.6\baselineskip}
\gasfile[firstline=5, lastline=36]{projects/sem10/switcher.s}

Заполните пропущенные части в~\lang{C} коде. Существует только один способ разместить варианты в~шаблоне, за~исключением порядка \code{case}-меток~\code{C} и~\code{D}.

\end{exercise}
