% !TEX encoding   = UTF8
% !TEX spellcheck = ru_RU
% !TEX root = ../seminars.tex

%%=============
\section{Циклы}
%%=============
Язык~\lang{C} предоставляет несколько различных циклов: \code{do-while}, \code{while} и \code{for}. В~машинном коде нет соответствующих инструкций. Вместо этого для~реализации таких конструкций используются комбинации условных и безусловных переходов. \GCC{} и другие компиляторы генерируют код для~реализации циклов, придерживаясь двух основных шаблонов, которые будут рассмотрены ниже.



%%=================================
\subsection{Цикл \texttt{do-while}}
%%=================================
Общая форма инструкции \code{do-while} в~\lang{C} задаётся шаблоном:

\begin{ccode*}{linenos=false}
do
  |\mystyle{body-statement}|
  while (|\mystyle{test-expr}|);
\end{ccode*}

\noindent Цикл выполняет тело \mystyle{body-statement}, затем вычисляет условие \mystyle{test-expr} и продолжает работу повторно, пока результатом вычисления условия является не~нуль. Эта форма может быть преобразована в~\code{goto}-версию следующим образом:

\begin{ccode*}{linenos=false}
loop:
  |\mystyle{body-statement}|
  t = |\mystyle{test-expr}|;
  if (t)
    goto loop;
\end{ccode*}



%%=====================
\paragraph{Упражнение.}
%%=====================
Дана~\lang{C} функция:

\cfile{projects/sem09/loop_do_while.c}

\noindent для~которой \GCC{} генерирует следующий ассемблерный код:

\precomment/x initially in %rdi/
\vspace{-1.7\baselineskip}
\gasfile[firstline=5, lastline=15]{projects/sem09/loop_do_while.s}

\begin{enumIssue}
  \item Какие регистры используются для~хранения значений переменных~\code{x}, \code{y} и~\code{n}?
  \item Каким образом компилятор устранил необходимость в~указателе~\code{p} и его разыменовании, которое подразумевает наличие в~коде выражения \code{(*p)+=5}?
  \item Поясните, как ассемблерная версия отображает код на~\lang{C}.
\end{enumIssue}



%%==============================
\subsection{Цикл \texttt{while}}
%%==============================
Общая форма инструкции \code{while} в~\lang{C} задаётся шаблоном:

\begin{ccode*}{linenos=false}
while (|\mystyle{test-expr}|)
  |\mystyle{body-statement}|
\end{ccode*}

\noindent Он отличается от~цикла \code{do-while} тем, что \mystyle{test-expr} вычисляется вначале, и цикл может закончится, не~выполнив \mystyle{body-statement} ни~разу. Существует много способов перевода цикла \code{while} в~машинный код, два из~которых используются в~коде, генерируемом \GCC. Оба способа используют ту же структуру, что и для~\code{do-while}, но отличаются реализацией начальной проверки условия.


%%===============================================
\paragraph{Первый способ <<переход в~середину>>.}
%%===============================================
\enlargethispage{\baselineskip}
\begin{ccode*}{linenos=false}
  goto test;
loop:
  |\mystyle{body-statement}|
test:
  t = |\mystyle{test-expr}|;
  if (t)
    goto loop;
\end{ccode*}


%%=====================
\paragraph{Упражнение.}
%%=====================
Ниже приведена общая структура кода на~\lang{C}:

{\newcommand*{\ans}{\ansfw{10em}}%
%
\begin{ccode*}{linenos=false}
long loop_while (long a, long b)
{
  long result = |\ans{0}|;
  while (|\ans{a > b}|)
  {
    result = |\ans{result + (a + b)}|;
    a = |\ans{a - 1}|;
  }
  return result;
}
\end{ccode*}
}

\noindent для~которой \GCC{} выдаёт следующий код:

\precomment/a: %rdi, b: %rsi/
\vspace{-1.7\baselineskip}
\gasfile[firstline=5, lastline=15]{projects/sem09/loop_while.s}

\noindent Видно, что компилятор использовал переход-в-середину (\textenglish{jump-to-middle}), применив инструкцию \code{jmp} в~строке~7 для~перехода к~проверке условия, которое начинается с~метки~\code{.L2}. Заполните пропущенные части в~\lang{C} коде.



%%=================================================
\paragraph{Второй способ <<защищённый-\code{do}>>.}
%%=================================================
Код преобразуется в~форму \code{do-while} и использует условный переход, чтобы пропустить выполнение тела, если условие сразу не~выполняется:

\begin{ccode*}{linenos=false}
  t = |\mystyle{test-expr}|;
  if (!t)
    goto done;
  do
    |\mystyle{body-statement}|
    while (|\mystyle{test-expr}|);
done:
\end{ccode*}



%%=====================
\paragraph{Упражнение.}
%%=====================
Ниже приведена общая структура кода на~\lang{C}:

{\newcommand*{\ans}{\ansfw{6.5em}}%
%
\begin{ccode*}{linenos=false}
long loop_while2 (long a, long b)
{
  long result = |\ans{b}|;
  while (|\ans{b > 0}|)
  {
    result = |\ans{result * a}|;
    b = |\ans{b - a}|;
  }
  return result;
}
\end{ccode*}
}

\enlargethispage{1.2\baselineskip}
\noindent для~которой \GCC{}, запущенный с~ключом \code{-O1}, выдаёт следующий код:
%
\precomment/a: %rdi, b: %rsi/
\vspace{-1.4\baselineskip}
\gasfile[firstline=5, lastline=17]{projects/sem09/loop_while2.s}

\noindent Видно, что компилятор использовал защищённый-\code{do} (\textenglish{guarded-do}), применив инструкцию \code{jle} в~строке~7 для~пропуска выполнения тела, если условие оказалось невыполненным. Заполните пропущенные части в~\lang{C} коде.



%%================
\WhatToReadSection
%%================
\citeauthor[глава~3, стр.~225--242]{Bryant:2022:ru}



%%===============
\ExercisesSection
%%===============
\begin{exercise}
\item Функция \code{fun\_a} имеет следующую структуру:

{\newcommand*{\ans}{\ansdots}%
%
\begin{ccode}
long fun_a (unsigned long x)
{
  long val = 0;

  while (|\ans{x}|)
  {
    |\ans{val \textasciicircum{}= x;}|
    |\ansx{x >>= 1;}|
  }

  return |\ans{val & 0x1}|;
}
\end{ccode}
}

\GCC{} генерирует для~этой функции ассемблерный код, показанный ниже:

\precomment/x: %rdi/
\vspace{-0.7\baselineskip}
\gasfile[firstline=5, lastline=15]{projects/sem09/fun_a.s}

Разберитесь, как работает этот код, и выполните следующее:

\begin{enumIssue}
\item Определите, какой способ трансляции циклов был использован.
\item При~помощи ассемблерного кода восстановите пропущенные части в~\lang{C}.
\item Опишите словами, что вычисляет функция.
\end{enumIssue}

\end{exercise}
