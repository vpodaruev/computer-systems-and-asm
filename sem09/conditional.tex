% !TEX encoding   = UTF8
% !TEX spellcheck = ru_RU
% !TEX root = ../seminars.tex

%%=========================
\chapter{Ветвления и циклы}
%%=========================

%%=====================================
\section{Реализация условных ветвлений}
%%=====================================
Общая форма конструкции \code{if-else} в~\lang{C} задаётся шаблоном:

\begin{ccode*}{linenos=false}
if (|\mystyle{test-expr}|)
  |\mystyle{then-statement}|
else
  |\mystyle{else-statement}|
\end{ccode*}

\noindent где \mystyle{test-expr} является целочисленным выражением, результатом которого является либо нуль (интерпретируется как \code{false}), либо ненулевое значение (интерпретируется как \code{true}). Выполняется только одна из~ветвей: \mystyle{then-statement} или \mystyle{else-statement}.

Эту форму ассемблер обычно реализует в~виде, который можно изобразить, используя синтаксис~\lang{C}, следующим образом:

\begin{ccode*}{linenos=false}
  t = |\mystyle{test-expr}|;
  if (!t)
    goto false;
  |\mystyle{then-statement}|
  goto done;
false:
  |\mystyle{else-statement}|
done:
\end{ccode*}



%%=====================
\paragraph{Упражнение.}
%%=====================
Дана~\lang{C} функция:

\cfile[linenos=false]{projects/sem09/if-else/cond.c}

\noindent для~которой \GCC{} генерирует следующий ассемблерный код:

\precomment/a: %rdi, p: %rsi/
\vspace{-1.6\baselineskip}
\gasfile[firstline=5, lastline=12]{projects/sem09/if-else/cond.s}

\begin{enumIssue}
  \item Напишите \code{goto}-версию на~\lang{C}, которая выполняет те же самые вычисления и отражает поток выполнения ассемблерного кода.

  \item Объясните, почему ассемблерный код содержит два условных ветвления, хотя в~исходном коде~\lang{C} только одна ветвь.
\end{enumIssue}



%%==============================
\section{Копирование по~условию}
%%==============================
Альтернативной стратегией для~реализации условных ветвлений является пересылка данных по~условию. В~этом подходе вычисляются выражения в~обеих ветвях условного оператора, а затем выбирается одно в~зависимости от~выполнения условия. Такая стратегия имеет смысл только в~ограниченных случаях, но она может быть реализована простой инструкцией \emph{копирования по~условию}, которая имеет более высокую производительность на~современных процессорах.

\begin{flushleft}\small\ttfamily\begin{tabular}{@{}llll>{\rmfamily}l@{}}
  \toprule
  \multicolumn{2}{@{}l}{\textrm{Инструкция}} & \textrm{Синоним} & \textrm{Условие} & Пояснение \\
  \midrule
  cmove   & S,R & cmovz  &                  ZF & равно/нуль \\
  cmovne  & S,R & cmovnz & \textasciitilde{}ZF & не~равно/не~нуль \\[0.5em]

  cmovs   & S,R &        &                  SF & отрицательное \\
  cmovns  & S,R &        & \textasciitilde{}SF & неотрицательное \\[0.5em]

  cmovg   & S,R & cmovnle & \textasciitilde{}(SF \textasciicircum{} OF) \& \textasciitilde{}ZF & больше (знаковое $>$) \\
  cmovge  & S,R & cmovnl  & \textasciitilde{}(SF \textasciicircum{} OF) & больше или равно (знаковое $\geqslant$) \\
  cmovl   & S,R & cmovnge & SF \textasciicircum{} OF                    & меньше (знаковое $<$) \\
  cmovle  & S,R & cmovng  & (SF \textasciicircum{} OF) | ZF             & меньше или равно (знаковое $\leqslant$) \\[0.5em]

  cmova   & S,R & cmovnbe & \textasciitilde{}CF \& \textasciitilde{}ZF & выше (беззнаковое $>$) \\
  cmovae  & S,R & cmovnb  & \textasciitilde{}СF                        & выше или равно (беззнаковое $\geqslant$) \\
  cmovb   & S,R & cmovnae & СF                                         & ниже (беззнаковое $<$) \\
  cmovbe  & S,R & cmovna  & CF | ZF                                    & ниже или равно (беззнаковое $\leqslant$) \\
  \bottomrule
\end{tabular}\end{flushleft}

Чтобы понять, как условные операции могут быть реализованы через пересылку данных, рассмотрим следующую общую форму условного выражения с~присваиванием:

\cc/v = |\mystyle{test-expr}| ? |\mystyle{then-expr}| : |\mystyle{else-expr}|;/

Стандартный способ скомпилировать это выражение, используя условные переходы, имеет следующий вид:

\begin{ccode*}{linenos=false}
  if (!|\mystyle{test-expr}|)
    goto false;
  v = |\mystyle{then-expr}|;
  goto done;
false:
  v = |\mystyle{else-expr}|;
done:
\end{ccode*}

Этот код содержит две последовательности команд "--- одну для~вычисления \mystyle{then-expr} и вторую для~вычисления \mystyle{else-expr}. Комбинация условных и безусловных переходов гарантирует, что будет выполнена только одна из~этих последовательностей.

Что касается кода, основанного на~пересылке по~условию, то необходимо вычислять оба выражения: \mystyle{then-expr} и \mystyle{else-expr}. Финальный результат выбирается, в~зависимости от~значения \mystyle{test-expr}:

\begin{ccode*}{linenos=false}
  v  = |\mystyle{then-expr}|;
  ve = |\mystyle{else-expr}|;
  t  = |\mystyle{test-expr}|;
  if (!t) v = ve;
\end{ccode*}

Последняя строка этого кода реализуется операцией пересылки по~условию: зачение~\code{ve} копируется в~\code{v}, только если условие~\code{t} не~выполняется.

Не~все условные выражения могут быть скомпилированы, используя пересылку данных по~условию. Если одно из~выражений условного оператора может вызывать ошибку или имеет побочные эффекты, такой код может привести к~ошибочному поведению:

\cc/long cread (long *xp)  { return xp ? *xp : 0; }/

Использование пересылки по~условию не~всегда повышает производительность кода. Компиляторы должны принимать во~внимание соотношение затрат на~ненужные вычисления и стоимости ошибки в~предсказании перехода (потери на~перезапуск конвейера).



%%=====================
\paragraph{Упражнение.}
%%=====================
Ниже приведена функция на~\lang{C}, в~которой определение операции~\code{OP} не~завершено:

\begin{ccode}
#define OP _____  /* Unknown operator */
long arith (long x)  { return x OP 8; }  |\ansx{\hfill\# Ответ: /}|
\end{ccode}

\noindent При~компиляции \GCC{} генерирует следующий ассемблерный код:

\precomment/x: %rdi/
\vspace{-1.6\baselineskip}
\gasfile[firstline=5, lastline=10]{projects/sem09/if-expr/arith.s}

\begin{enumIssue}
  \item Что использовано в~качестве операции~\code{OP}?
  \item Объясните, как работает этот код.
\end{enumIssue}



%%===============
\ExercisesSection
%%===============
\begin{exercise}

\item Компилируя код на~\lang{C} вида:
{\newcommand*{\ans}{\ansfw{5.5em}}%
\begin{ccode}
long test (long x, long y, long z)
{
  long val = |\ans{z + y - x}|;

  if (|\ans{z > 5}|)
  {
    if (|\ans{y > 2}|)
      val = |\ans{x * z}|;
    else
      val = |\ans{x * y}|;
  }
  else if (|\ans{z < 3}|)
    val = |\ans{y * z}|;

  return val;
}
\end{ccode}
}
\GCC{} генерирует следующий ассемблерный листинг:

\precomment/x: %rdi, y: %rsi, z: %rdx/
\vspace{-0.6\baselineskip}
\gasfile[firstline=5, lastline=25]{projects/sem09/if-else/test.s}

Заполните пропущенные в~\lang{C} коде выражения.



\item Компилируя код на~\lang{C} вида:
{\newcommand*{\ans}{\ansfw{4.5em}}%
\begin{ccode}
long test (long x, long y)
{
  long val = |\ans{12 + y}|;

  if (|\ans{x < 0}|)
  {
    if (|\ans{x < y}|)
      val = |\ans{y - x}|;
    else
      val = |\ans{y \& x}|;
  }
  else if (|\ans{y > 10}|)
    val = |\ans{x + y}|;

  return val;
}
\end{ccode}
}
\GCC{}, запущенный с~ключом \code{-O1}, генерирует следующий ассемблерный листинг:\enlargethispage{1\baselineskip}

\precomment/x: %rdi, y: %rsi/
\vspace{-0.6\baselineskip}
% asmlst -fif-conversion test.c
\gasfile[firstline=5, lastline=20]{projects/sem09/if-expr/test.s}

Заполните пропущенные в~\lang{C} коде выражения.

\end{exercise}



%%=================
% !TEX encoding   = UTF8
% !TEX spellcheck = ru_RU
% !TEX root = ../seminars.tex

%%=============
\section{Циклы}
%%=============
Язык~\lang{C} предоставляет несколько различных циклов: \code{do-while}, \code{while} и \code{for}. В~машинном коде нет соответствующих инструкций. Вместо этого для~реализации таких конструкций используются комбинации условных и безусловных переходов. \GCC{} и другие компиляторы генерируют код для~реализации циклов, придерживаясь двух основных шаблонов, которые будут рассмотрены ниже.



%%=================================
\subsection{Цикл \texttt{do-while}}
%%=================================
Общая форма инструкции \code{do-while} в~\lang{C} задаётся шаблоном:

\begin{ccode*}{linenos=false}
do
  |\mystyle{body-statement}|
  while (|\mystyle{test-expr}|);
\end{ccode*}

\noindent Цикл выполняет тело \mystyle{body-statement}, затем вычисляет условие \mystyle{test-expr} и продолжает работу повторно, пока результатом вычисления условия является не~нуль. Эта форма может быть преобразована в~\code{goto}-версию следующим образом:

\begin{ccode*}{linenos=false}
loop:
  |\mystyle{body-statement}|
  t = |\mystyle{test-expr}|;
  if (t)
    goto loop;
\end{ccode*}



%%=====================
\paragraph{Упражнение.}
%%=====================
Дана~\lang{C} функция:

\cfile{projects/sem09/loop_do_while.c}

\noindent для~которой \GCC{} генерирует следующий ассемблерный код:

\precomment/x initially in %rdi/
\vspace{-1.7\baselineskip}
\gasfile[firstline=5, lastline=15]{projects/sem09/loop_do_while.s}

\begin{enumIssue}
  \item Какие регистры используются для~хранения значений переменных~\code{x}, \code{y} и~\code{n}?
  \item Каким образом компилятор устранил необходимость в~указателе~\code{p} и его разыменовании, которое подразумевает наличие в~коде выражения \code{(*p)+=5}?
  \item Поясните, как ассемблерная версия отображает код на~\lang{C}.
\end{enumIssue}



%%==============================
\subsection{Цикл \texttt{while}}
%%==============================
Общая форма инструкции \code{while} в~\lang{C} задаётся шаблоном:

\begin{ccode*}{linenos=false}
while (|\mystyle{test-expr}|)
  |\mystyle{body-statement}|
\end{ccode*}

\noindent Он отличается от~цикла \code{do-while} тем, что \mystyle{test-expr} вычисляется вначале, и цикл может закончится, не~выполнив \mystyle{body-statement} ни~разу. Существует много способов перевода цикла \code{while} в~машинный код, два из~которых используются в~коде, генерируемом \GCC. Оба способа используют ту же структуру, что и для~\code{do-while}, но отличаются реализацией начальной проверки условия.


%%===============================================
\paragraph{Первый способ <<переход в~середину>>.}
%%===============================================
\enlargethispage{\baselineskip}
\begin{ccode*}{linenos=false}
  goto test;
loop:
  |\mystyle{body-statement}|
test:
  t = |\mystyle{test-expr}|;
  if (t)
    goto loop;
\end{ccode*}


%%=====================
\paragraph{Упражнение.}
%%=====================
Ниже приведена общая структура кода на~\lang{C}:

{\newcommand*{\ans}{\ansfw{10em}}%
%
\begin{ccode*}{linenos=false}
long loop_while (long a, long b)
{
  long result = |\ans{0}|;
  while (|\ans{a > b}|)
  {
    result = |\ans{result + (a + b)}|;
    a = |\ans{a - 1}|;
  }
  return result;
}
\end{ccode*}
}

\noindent для~которой \GCC{} выдаёт следующий код:

\precomment/a: %rdi, b: %rsi/
\vspace{-1.7\baselineskip}
\gasfile[firstline=5, lastline=15]{projects/sem09/loop_while.s}

\noindent Видно, что компилятор использовал переход-в-середину (\textenglish{jump-to-middle}), применив инструкцию \code{jmp} в~строке~7 для~перехода к~проверке условия, которое начинается с~метки~\code{.L2}. Заполните пропущенные части в~\lang{C} коде.



%%=================================================
\paragraph{Второй способ <<защищённый-\code{do}>>.}
%%=================================================
Код преобразуется в~форму \code{do-while} и использует условный переход, чтобы пропустить выполнение тела, если условие сразу не~выполняется:

\begin{ccode*}{linenos=false}
  t = |\mystyle{test-expr}|;
  if (!t)
    goto done;
  do
    |\mystyle{body-statement}|
    while (|\mystyle{test-expr}|);
done:
\end{ccode*}



%%=====================
\paragraph{Упражнение.}
%%=====================
Ниже приведена общая структура кода на~\lang{C}:

{\newcommand*{\ans}{\ansfw{6.5em}}%
%
\begin{ccode*}{linenos=false}
long loop_while2 (long a, long b)
{
  long result = |\ans{b}|;
  while (|\ans{b > 0}|)
  {
    result = |\ans{result * a}|;
    b = |\ans{b - a}|;
  }
  return result;
}
\end{ccode*}
}

\enlargethispage{1.2\baselineskip}
\noindent для~которой \GCC{}, запущенный с~ключом \code{-O1}, выдаёт следующий код:
%
\precomment/a: %rdi, b: %rsi/
\vspace{-1.4\baselineskip}
\gasfile[firstline=5, lastline=17]{projects/sem09/loop_while2.s}

\noindent Видно, что компилятор использовал защищённый-\code{do} (\textenglish{guarded-do}), применив инструкцию \code{jle} в~строке~7 для~пропуска выполнения тела, если условие оказалось невыполненным. Заполните пропущенные части в~\lang{C} коде.



%%================
\WhatToReadSection
%%================
\citeauthor[глава~3, стр.~225--242]{Bryant:2022:ru}



%%===============
\ExercisesSection
%%===============
\begin{exercise}
\item Функция \code{fun\_a} имеет следующую структуру:

{\newcommand*{\ans}{\ansdots}%
%
\begin{ccode}
long fun_a (unsigned long x)
{
  long val = 0;

  while (|\ans{x}|)
  {
    |\ans{val \textasciicircum{}= x;}|
    |\ansx{x >>= 1;}|
  }

  return |\ans{val & 0x1}|;
}
\end{ccode}
}

\GCC{} генерирует для~этой функции ассемблерный код, показанный ниже:

\precomment/x: %rdi/
\vspace{-0.7\baselineskip}
\gasfile[firstline=5, lastline=15]{projects/sem09/fun_a.s}

Разберитесь, как работает этот код, и выполните следующее:

\begin{enumIssue}
\item Определите, какой способ трансляции циклов был использован.
\item При~помощи ассемблерного кода восстановите пропущенные части в~\lang{C}.
\item Опишите словами, что вычисляет функция.
\end{enumIssue}

\end{exercise}

%%=================
