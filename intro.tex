% !TEX encoding   = UTF8
% !TEX spellcheck = ru_RU
% !TEX root = ../../seminars.tex

%%=====================
\section{Цели и задачи}
%%=====================
Посмотреть, во что отображается программа на языке~\lang{C} при переходе на более низкий уровень~[абстракции], какие действия выполняет компилятор в процессе сборки исполняемого модуля, что умеет непосредственно сам компьютер.

Познакомиться с важными аспектами современной архитектуры, которые необходимо учитывать при написании программ.

Архитектура \name{Intel}\,\name{x86-64}, синтаксис языка ассемблера \name{AT\&T}.



%%==================
\section{Литература}
%%==================

%%===================
\paragraph{Основная:}
%%===================
\begin{enumerate}
  \item \cite{Harris:2015:ru}
  \item \cite{Bryant:2016:en}
\end{enumerate}



%%=========================
\paragraph{Дополнительная:}
%%=========================
\begin{enumerate}[resume]
  \item \cite{Tanenbaum:2013:ru}
  \item \cite{Vorozhcov:2008:ru}
  \item \cite{Pacheco:2011:en}
\end{enumerate}
\nocite{WikiBookAsm:ru, Zubkov:2000:ru}



%%===========================
\section{Материалы и задания}
%%===========================
Будут размещены по мере необходимости на \href{\yadiskurl}{яндекс-диске\footnote{\nolinkurl{\yadiskurl}}} в каталогах:
\begin{itemfeature}
  \item \codebf{books} "--- основная и дополнительная литература;
  \item \codebf{asm-lectures} "--- лекционные материалы;
  \item \codebf{asm-seminars} "--- семинарские материалы;
  \begin{itemize}
    \item \codebf{/program.pdf} "--- программа курса, темы зачёта;
    \item \codebf{/progress.pdf} "--- планы, успеваемость и проверочные мероприятия;
    \item \codebf{/seminars.pdf} "--- вспомогательная методичка по материалам занятий.
  \end{itemize}
\end{itemfeature}



%%===============================
\section{Программное обеспечение}
%%===============================
\begin{itemfeature}
  \item текстовый редактор,
  \item компиляторы языков \lang{C}/\lang{C++},
  \item ассемблер для архитектуры \name{Intel}\,\name{x86-64}, поддерживающий синтаксис \name{AT\&T},
  \item отладчик \name{gdb},
  \item утилита \name{objdump}.
\end{itemfeature}

Инструкции по установке и настройке рабочей среды размещены в приложении на странице~\pageref{sect:workEnv}.
