% !TEX encoding   = UTF8
% !TEX spellcheck = ru_RU
% !TEX root = ../seminars.tex

%%==================================
\chapter{Массивы и структуры данных}
%%==================================

%%=====================================
\section{Массивы и адресная арифметика}
%%=====================================

\cc`T A[N];`

Обозначим через~\(x_A\) адрес начала массива. Во-первых, объявление выделяет непрерывную область памяти размером \(L\cdot N\) байт, где~\(L\) "--- размер (в~байтах) данных типа~\(T\). Во-вторых, оно вводит в~область видимости идентификатор~\(A\), который может использоваться как указатель на~начало массива. Доступ к~элементам осуществляется по~индексу в~диапазоне от~\(0\) до~\(N-1\). Элемент с~индексом~\(i\) размещается по~адресу \(x_A + L\cdot i\).



%%=====================
\paragraph{Упражнение.}
%%=====================
Рассмотрим следующие объявления:

\begin{ccode}
int P[5];
short Q[2];
int **R[9];
double *S[10];
short *T[2];
\end{ccode}

Заполните таблицу, описывающую размер элементов, суммарный размер и адрес \(i\)-го элемента для~каждого массива.

\begin{flushleft}\newcommand*{\ans}[1]{\ansfw{3cm}{#1}}%
\texttt\small
\begin{tabular}{@{}lllll@{}}
  \textrm{Массив} & \textrm{Размер элемента} & \textrm{Суммарный размер} & \textrm{Адрес начала} & \textrm{Адрес \(i\)-го элемента} \\
\midrule
  P & \ans{4} & \ans{20} & \(x_P\) & \ans{\(x_P + 4i\)} \\
  Q & \ans{2} & \ans{ 4} & \(x_Q\) & \ans{\(x_Q + 2i\)} \\
  R & \ans{8} & \ans{72} & \(x_R\) & \ans{\(x_R + 8i\)} \\
  S & \ans{8} & \ans{80} & \(x_S\) & \ans{\(x_S + 8i\)} \\
  T & \ans{8} & \ans{16} & \(x_T\) & \ans{\(x_T + 8i\)} \\
\end{tabular}\end{flushleft}



%%=====================
\paragraph{Упражнение.}
%%=====================
Положим, что~\(x_P\) "--- адрес массива~\(P\) типа \code{short} и индекс~\(i\) загружены в~регистры \code{\%rdx} и \code{\%rcx}, соответственно. Для~каждого из~следующих выражений выпишите тип, формулу, определяющую значение, и ассемблерную реализацию. Результат должен быть записан в~регистр \code{\%rax} (или его часть \code{\%ax}).

\begin{flushleft}
\newcommand*{\anst}[1]{\ansfw{1.6cm}{#1}}%
\newcommand*{\ansv}[1]{\ansfw{3cm}{#1}}%
\newcommand*{\ansa}[1]{\ansfw{7cm}{#1}}%
\texttt\small
\begin{tabular}{@{}llll@{}}
  \textrm{Выражение} & \textrm{Тип} & \textrm{Значение} & \textrm{Ассемблерный код} \\
\midrule
  P[1]       & \anst{short}  & \ansv{M[\(x_P + 2\)]}       & \ansa{movw 2(\%rdx), \%ax} \\
  P + 3 + i  & \anst{short*} & \ansv{\(x_P + 2i + 6\)}     & \ansa{leaq 6(\%rdx,\%rcx,2), \%rax} \\
  P[i*4 - 5] & \anst{short}  & \ansv{M[\(x_P + 8i - 10\)]} & \ansa{movw -10(\%rdx,\%rcx,8), \%ax} \\
  P[2]       & \anst{short}  & \ansv{M[\(x_P + 4\)]}       & \ansa{movw 4(\%rdx), \%ax} \\
  \&P[i+2]   & \anst{short*} & \ansv{\(x_P + 2i + 4\)}     & \ansa{leaq 4(\%rdx,\%rcx,2), \%rax} \\
\end{tabular}\end{flushleft}



%%=====================
\paragraph{Упражнение.}
%%=====================
Приведённая ниже функция транспонирует элементы массива \(M\times M\), где \(M\)~определяется при~помощи \code{\#define}:

\cfile[firstline=3]{projects/sem12/transpose.c}

\noindent При~оптимизации с~ключом~\code{-O1} \GCC{} генерирует для~внутреннего цикла следующий код:

\gasfile[firstline=20, lastline=28]{projects/sem12/transpose.s}

\noindent Очевидно, что обращение к~элементам по~индексу \GCC{} преобразовал в~код с~указателями.

\begin{enumIssue}
\item Какой регистр содержит указатель на~элемент массива \code{A[i][j]}?
\item Какой регистр содержит указатель на~элемент массива \code{A[j][i]}?
\item Чему равно значение~\(M\)?
\end{enumIssue}



%%=================
\section{Структуры}
%%=================
Объявление \code{struct} в~\lang{C} создаёт тип данных, который группирует объекты возможно различных типов в~один объект. Доступ к~различным компонентам структуры осуществляется по~имени. Реализация структур подобна реализации массивов в~том, что все компоненты структуры размещаются в~непрерывном участке памяти и указатель на~структуру хранит адрес её первого байта. Компилятор сохраняет информацию о~каждом типе-структуре, запоминая смещения в~байтах каждого поля. Он генерирует ссылки на~элементы структуры, используя эти смещения в~качестве смещений для~инструкций обращения в~память.

В~качестве примера рассмотрим следующее объявление:

\cfile[lastline=7]{projects/sem12/record.c}

\begin{flushleft}
\newcommand*{\col}[2]{\multicolumn{1}{#1}{\scriptsize #2}}
\arrayrulewidth=1pt
\texttt\small
\begin{tabular}{r|c|c|c|c|c|l}
  \col{r}{Offset} & \col{p{2cm}}{0} & \col{p{2cm}}{4} & \col{p{2cm}}{8} & \col{p{2cm}}{} & \col{p{4cm}}{16} & \col{l}{24} \\[-0.1em]
\cline{2-6}
  \scriptsize Contents &\rule{0pt}{1em} i & j & a[0] & a[1] & p & \\
\cline{2-6}
\end{tabular}\end{flushleft}

\noindent и реализацию инструкции:

\cfile[firstline=12, lastline=12]{projects/sem12/record.c}

\noindent полагая, что в~начальный момент значение переменной~\code{r} типа \code{Record*} записано в~регистр \code{\%rdi}:

\gasfile[firstline=6, lastline=10]{projects/sem12/record.s}



%%=====================
\paragraph{Упражнение.}
%%=====================
Рассмотрим объявление структуры:

\cfile[lastline=10]{projects/sem12/st_init.c}

Это объявление показывает, что одна структура может быть встроена в~другую так же, как массивы могут быть встроены внутрь структур и массивов [большей размерности].

Следующая ниже процедура (с~некоторыми опущенными выражениями) выполняет некоторые операции над~приведённой выше структурой:

{\newcommand*{\ans}{\ansfw{5em}}%
%
\begin{ccode*}{linenos=false}
void st_init (struct Test *st)
{
  st->s.y  = |\ans{st.s.x}|;
  st->p    = |\ans{\&st->s.y}|;
  st->next = |\ans{st}|;
}
\end{ccode*}
}

\begin{enumIssue}
  \item Чему равны смещения (в~байтах) следующих полей?

  \begin{flushleft}
  \newcommand*{\ans}{\ansfw{3em}}%
  \texttt\small
  \begin{tabular}{r@{:\quad}l}
    p    & \ans{ 0} \\
    s.x  & \ans{ 8} \\
    s.y  & \ans{12} \\
    next & \ans{16} \\
  \end{tabular}
  \end{flushleft}

  \item Сколько байтов всего занимает экземпляр структуры?

  \item Компилятор генерирует следующий ассемблерный код:

  \gasfile[firstline=5, lastline=11]{projects/sem12/st_init.s}

  На~основе этой информации заполните пропущенные выражения в~коде функции \code{st\_init()}.
\end{enumIssue}



%%====================
\section{Выравнивание}
%%====================
Многие компьютерные системы накладывают ограничения на~допустимые значения адресов для~примитивных (базовых) типов данных, требуя, чтобы адрес объекта был кратен некоторому значению~\(K\) (как правило, \(2\), \(4\) или~\(8\)). Такое \emph{ограничение на~выравнивание} упрощает проектирование аппаратного обеспечения, образующего интерфейс между процессором и системой памяти.

Аппаратное обеспечение \name{x86-64} будет работать корректно вне зависимости от~выравнивания данных. Тем не~менее, \name{Intel} рекомендует выравнивать данные, чтобы улучшить производительность системы памяти. Правило выравнивания основывается на~принципе, что любой примитивный объект, занимающий~\(K\) байт, должен иметь адрес, кратный~\(K\). Из~этого правила вытекает следующая таблица выравниваний:

\begin{flushleft}\small\begin{tabular}{l>{\ttfamily}l}
  \(K\) & \textrm{Типы} \\
\midrule
  \(1\) & char \\
  \(2\) & short \\
  \(4\) & int, float \\
  \(8\) & long, double, char* \\
\end{tabular}\end{flushleft}

Выравнивание осуществляется путём проверки, что каждый тип данных устроен таким образом, что все объекты внутри него удовлетворяют своим ограничениям на~выравнивание. Компилятор размещает директивы в~ассемблерном коде, указывая требуемое выравнивание для~глобальных данных. Например, для~таблицы переходов оператора \code{switch} на~странице \pageref{ex:jumptable}:

\gas`.align 8`

При~работе со~структурами компилятор может выделять больше памяти и оставлять между полями промежутки, чтобы гарантировать, что каждый элемент удовлетворяет своим ограничениям на~выравнивание. Рассмотрим пример:

\begin{ccode*}{linenos=false}
struct S1
{
  int i;
  char c;
  int j;
};
\end{ccode*}

\begin{flushleft}
\newcommand*{\col}[2]{\multicolumn{1}{#1}{\scriptsize #2}}
\arrayrulewidth=1pt
\texttt\small
\begin{tabular}{r|c|c|c|c|l}
  \col{r}{Offset} & \col{p{2cm}}{0} & \col{p{0.5cm}}{4} & \col{p{1.5cm}}{5} & \col{p{2cm}}{8} & \col{l}{12} \\[-0.1em]
\cline{2-5}
  \scriptsize Contents &\rule{0pt}{1em} i & c & \color{cyan} gap & j & \\
\cline{2-5}
\end{tabular}\end{flushleft}

Экземпляр структуры также имеет ограничение на~выравнивание начального адреса. Например, компилятор может добавить промежуток в~конце, чтобы удовлетворить ограничение на~выравнивание элементов в~массиве:

\begin{ccode*}{linenos=false}
struct S2
{
  int i;
  int j;
  char c;
} d[4];
\end{ccode*}

\begin{flushleft}
\newcommand*{\col}[2]{\multicolumn{1}{#1}{\scriptsize #2}}
\arrayrulewidth=1pt
\texttt\small
\begin{tabular}{r|c|c|c|c|l}
  \col{r}{Offset} & \col{p{2cm}}{0} & \col{p{2cm}}{4} & \col{p{0.5cm}}{8} & \col{p{1.5cm}}{9} & \col{l}{12} \\[-0.1em]
  \cline{2-5}
  \scriptsize Contents &\rule{0pt}{1em} i & j & c & \color{cyan} gap & \\
  \cline{2-5}
\end{tabular}\end{flushleft}



%%=====================
\paragraph{Упражнение.}
%%=====================
Для~каждого из~объявлений структуры, определите смещение всех полей, суммарный размер и требование к~выравниванию в~архитектуре \name{x86-64}:

\begin{enumIssue}
  \item \cinline`struct P1 { short i; int c; int *j; short *d; };`
  \item \cinline`struct P2 { int i[2]; char c[8]; short s[4]; long *j; };`
  \item \cinline`struct P3 { long w[2]; int *c[2]; };`
  \item \cinline`struct P4 { char w[16]; char *c[2]; };`
  \item \cinline`struct P5 { struct P4 a[2]; struct P1 t; };`
\end{enumIssue}



%%=========================================================
\HardSection{Выход за~границы памяти и переполнение буфера}
%%=========================================================
Напомним, что ни~язык~\lang{C}, ни~\lang{C++} не~выполняют никаких проверок выхода за~границы массива, а локальные переменные размещаются в~стеке наряду с~\emph{записью активации} "--- информацией о~состоянии программы, такой как значение регистров и адрес возврата. Эта комбинация может привести к~серьёзным ошибкам в~программе, если состояние, сохранённое в~стеке, будет повреждено записью в~<<элемент>> вне массива. Иными словами, когда программа попытается восстановить значение регистра или выполнить инструкцию \code{ret} при~повреждении состояния, дальнейшее выполнение может привести к~серьёзным ошибкам.

Очень распространённый источник повреждения состояния программы известен как \emph{переполнение буфера} (\textenglish{buffer overflow}). При~этом, как правило, создаётся локальный символьный массив для~хранения строки, но размер последней превышает выделенную память. Рассмотрим следующий пример:

\cfile{projects/sem12/echo.c}

Этот код показывает реализацию библиотечной функции \code{gets}, чтобы выявить серьёзную проблему, связанную с~ней. Она читает символы из~стандартного потока ввода в~\code{s} и останавливается, только когда встречает переход на~новую строку или происходит ошибка ввода. Затем завершает считанную строку нулевым байтом. Функция \code{echo} показывает пример использования \code{gets}.

Проблема заключается в~том, что не~существует способа определить, достаточно ли памяти выделено, чтобы вместить считываемую строку.

Исследуя сгенерированный ассемблерный код, можно выяснить структуру кадра стека функции \code{echo}:

\gasfile[firstline=42, lastline=52]{projects/sem12/echo.s} % -fno-stack-protector

По~мере увеличения строки ввода, следующая информация окажется поврежденной:

\begin{flushleft}\begin{tabular}{ll}
  Количество введённых символов & Дополнительное повреждённое состояние \\
\midrule
  \(0-7\)   & ничего \\
  \(8-15\)  & состояние вызывающей функции, регистр \code{\%rbx} \\
  \(16-23\) & адрес возврата \\
  \(24+\)   & состояние вызывающей функции \\
\end{tabular}\end{flushleft}

В~общем случае, использование \code{gets} или любой другой функции, которая может переполнить память, считается плохой практикой программирования. К~несчастью, некоторые часто используемые библиотечные функции, включая \code{strcpy}, \code{strcat} и \code{sprintf}, генерируют последовательность символов в~отсутствие информации о~размере целевого буфера. Такие условия порождают уязвимость к~переполнению буфера.

Более опасное использование переполнения буфера "--- дать программе выполнить нежелательную функцию. Это один из~распространённых приёмов взломать безопасность системы по~сети. Обычно программе скармливается строка, которая содержит выполнимый код, называемый \emph{вредоносным кодом} (\textenglish{exploit code}), и несколько дополнительных байтов, перезаписывающих адрес возврата указателем на~этот код. Выполнение инструкции \code{ret} вызывает переход к~выполнению вредоносного кода.



%%=========================================================
\HardSection{Способы защиты от~атак с~переполнением буфера}
%%=========================================================
Атаки с~переполнением буфера стали настолько распространёнными и вызывали так много проблем с~компьютерными системами, что современные компиляторы и операционные системы внедрили механизмы, чтобы усложнить проведение таких атак и ограничить способы, которыми злоумышленник может захватить управление системой через атаки с~переполнением буфера. Ниже представлены механизмы предоставляемые современными версиями \GCC{} под~\name{Linux}.



%%=============================
\paragraph{Рандомизация стека.}
%%=============================
Чтобы внедрить вредоносный код в~систему, взломщику необходимо вставить не~только сам код, но и указатель на~него как часть заражённой строки. Создание такого указателя требует знания адреса в~стеке, где будет расположена строка. Исторически, адреса в~программном стеке были в~высокой степени предсказуемыми.

Идея \emph{рандомизации стека} заключается в~его размещении по~различным адресам от~одного запуска программы к~другому. Таким образом, даже если множество машин запускают идентичный код, все они будут использовать различные адреса в~стеке. Такое поведение достигается за~счёт выделения случайного количества памяти в~стеке в~промежутке от~\(0\) до~\(n\) при~старте программы, например, при~помощи функции \code{alloca}, которая выделяет указанное ей количество памяти в~стеке. Выделенная память программой не~используется, но вызывает смещение всех последующих адресов в~стеке от~запуска к~запуску. Диапазон выделяемой памяти должен быть достаточно велик, чтобы получить существенные изменения адресов, но, в~то же время, и достаточно мал, чтобы не~расходовать память программы впустую.

Следующий код демонстрирует простой способ определения <<типичного>> адреса в~стеке (\code{stack\_address.cpp}):

\cppfile{projects/sem12/stack_address.cpp}

Собрать и запустить (10~раз) программу можно следующим образом:

\begin{consolecode}
$ g++ -o prog -Og stack_address.cpp
$ for (( i = 0; $i < 10; ++i )) do ./prog; done
stack address is 0x7ffcaf551740
stack address is 0x7fff203c31f0
stack address is 0x7ffdf1b0aea0
stack address is 0x7ffd2a5c0270
stack address is 0x7fff0067d450
stack address is 0x7ffcdc471290
stack address is 0x7fff63c3ff20
stack address is 0x7ffd0b84a910
stack address is 0x7ffd29d81c90
stack address is 0x7fffaae0b310
\end{consolecode}

Рандомизация стека стала стандартной практикой в~системах \name{Linux}. Она принадлежит более широкому классу приёмов, известному как \emph{рандомизация разметки адресного пространства} (\textenglish{ASLR "--- Address-Space Layout Randomization}). С~\textenglish{ASLR} различные части программы, включая код, библиотеки, стек, глобальные переменные и кучу, при~каждом запуске загружаются в~различные области памяти. Это может предотвратить некоторые формы атак, хотя и не~может гарантировать полной безопасности.

Аналогично примеру со~стеком, можно определить <<типичный>> адрес в~куче:

\cppfile[firstline=6, lastline=7]{projects/sem12/heap_address.cpp}

\noindent и в~сегменте кода:

\cppfile[firstline=5, lastline=6]{projects/sem12/code_address.cpp}



%%========================================
\paragraph{Обнаружение повреждения стека.}
%%========================================
На~второй линии защиты стоит механизм обнаружения, что стек был испорчен. Повреждение стека обычно происходит, когда программа выходит за~границу локального буфера. В~языке~\lang{C}, а также в~\lang{С++}, нет надёжного способа предотвращения записи вне границ массива. Вместо этого, программа может попытаться обнаружить, когда такая запись произошла, прежде, чем это нанесёт какой-либо вред.

Современные версии \GCC{} встраивают механизм, известный как \emph{защита стека} (\textenglish{stack protector}), в~генерируемый код, чтобы обнаружить переполнение буфера. Идея заключается в~размещении специального \emph{канареечного\footnote
{
  Термин <<канареечный>> сложился по~историческим причинам, поскольку этих птиц использовали для~обнаружения присутствия опасных газов в~угольных шахтах.
}
значения} (\textenglish{canary value, guard value}) в~кадре стека между локальным буфером и его остальной частью. Это значение генерируется случайным образом каждый раз при~запуске программы, так что не~существует простого способа для~взломщика определить его.

\GCC{} пытается выяснить уязвима ли функция к~переполнению буфера и вставляет код, реализующий описанную выше стратегию. Отключить такое поведение можно, задав ключ \code{-fno-stack-protector}.



%%===================================================
\paragraph{Ограничение областей с~исполняемым кодом.}
%%===================================================
Заключительный шаг "--- устранить способность взломщика встроить выполняемый код в~систему. Одним из~приёмов является ограничение областей памяти, которые могут выполняться.

В~типичных программах только часть памяти содержит код, сгенерированный компилятором, который необходимо исполнять. Остальная память может быть ограничена только чтением и записью.

Пространство виртуальной памяти логически разделено на~\emph{страницы}, обычно по~\(2048\) или \(4096\) байт. Аппаратное обеспечение поддерживает различные формы \emph{защиты памяти}, предоставляя права доступа к~этим страницам как для~пользовательских программ, так и для~ядра операционной системы. Многие современные системы поддерживают три формы доступа: чтение (чтение данных из~памяти), запись (запись данных в~память) и выполнение (трактовка содержимого памяти как машинного кода). Таким образом, стек может иметь права на~чтение и запись, но не~иметь прав на~выполнение. Проверка, может ли страница выполняться или нет, осуществляется аппартным обеспечением без~потери эффективности.



%%================
\WhatToReadSection
%%================
\citeauthor[глава~3, стр.~264--271, 273--276, 279--284, \(^*\)284--295]{Bryant:2022:ru}


%%===============
\ExercisesSection
%%===============
\begin{enumerate}
\item Ниже объявлена структура \code{ACE} и приведён прототип функции \code{test}:

\cfile[lastline=7]{projects/sem12/ace.c}

При~компиляции функции, \GCC{} выдаёт следующий ассемблерный листинг:

\gasfile[firstline=5, lastline=14]{projects/sem12/ace.s}

\begin{enumIssue}
\item Восстановите код на~\lang{C} для~функции \code{test}.
\item Опишите, какую структуру данных реализует \code{ACE} и какая операция выполняется функцией \code{test}.
\end{enumIssue}

\item Ответьте на~следующие вопросы для~данного объявления структуры:

\begin{ccode*}{linenos=false}
struct
{
  int    *a;
  float  b;
  char   c;
  short  d;
  long   e
  double f;
  int    g;
  char   *h;
} rec;
\end{ccode*}

\begin{enumIssue}
\item Каковы смещения (в~байтах) всех полей в~структуре?
\item Каков суммарный размер структуры?
\item Перегруппируйте поля так, чтобы минимизировать пустое пространство, и затем ещё раз ответьте на~вопросы для~полученной структуры.
\end{enumIssue}


\item Рассмотрите следующий код, где~\(R\), \(S\) и~\(T\) константы, объявленные при~помощи \code{\#define};

\cfile[firstline=6]{projects/sem12/store_ele.c}

При~компиляции этой функции с~ключом \code{-fno-pie} \GCC{} генерирует код:

\gasfile[firstline=5, lastline=17]{projects/sem12/store_ele.s}

\begin{enumIssue}
\item Выпишите формулу для~вычисления одномерного индекса, определяющую положение элемента \code{A[i][j][k]} в~массиве.
\item Основываясь на~приведённом выше ассемблерном коде, определите значения~\(R\), \(S\) и~\(T\).
\end{enumIssue}

\end{enumerate}
