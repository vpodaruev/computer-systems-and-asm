% !TEX encoding   = UTF8
% !TEX spellcheck = ru_RU
% !TEX root = ../seminars.tex

%%=======================
\chapter{Доступ к~данным}
%%=======================

%%============================================
\section{Структура модуля на~языке ассемблера}
%%============================================
\begin{gascode*}{linenos=false, frame=single}
  .globl   var       # declare var to be visible outside the module
  .bss               # specific [data] section: better-save-space
var:                 # a [label]
  .zero    4         # store 4 bytes, filled by zeroes

  .data              # [data] section
var2:
  .byte    'a'       # store byte, filled with ASCII code of symbol 'a'
var3:
  .long    4         # store 4 bytes, filled with an integer 4

  .text              # [code] section
func:
  mnemonic [op1[, op2[, op3]]]  # an [instruction] format
  # ...
                     # an [empty line] at the end
\end{gascode*}



%%================
\section{Регистры}\label{sect:registers}
%%================
{% registers
\newcommand{\bits}{%
  \tiny\begin{tabular}{p{6cm}p{3cm}p{1.5cm}@{}@{}p{1.5cm}}
    63\hfill & 31 & 15\hfill & ~7\hfill 0 \\
  \end{tabular}}%
%
\newcommand{\cpuReg}[5]{% CPU register
  \begin{tabular}{|p{6cm}|p{3cm}|p{1.5cm}@{}|@{}p{1.5cm}|l}
    \cline{1-4}
    \rule{0pt}{12pt}%
    #1\hfill & #2\hfill & #3 & ~\hfil#4\hfil & \sffamily --\hspace{0.5em}\slshape #5 \\
    \cline{1-4}
  \end{tabular}}%
%
\newcommand{\cpuRegGPe}[2]{% i386 general-purpose CPU register
  \cpuReg{\%r#1x}{\%e#1x}{\%#1x}{\%#1l}{#2}%
}%
\newcommand{\cpuRegSPe}[2]{% i386 specific-purpose CPU register
  \cpuReg{\%r#1}{\%e#1}{\%#1}{\%#1l}{#2}%
}%
\newcommand{\cpuRegGPr}[2]{% x86-64 general-purpose CPU register
  \cpuReg{\%r#1}{\%r#1d}{\%r#1w}{\%r#1b}{#2}%
}%
\newcommand{\cpuRegSPr}[2]{% x86-64 specific-purpose CPU register
  \cpuReg{\%r#1}{\%e#1}{\%#1}{}{#2}%
}%
\ttfamily\small\begin{tabbing}
  \hspace{-0.5cm}\=\bits \\
  \>\cpuRegGPe{a}{accumulator, return value} \\[0.2em]
  \>\cpuRegGPe{b}{base, callee saved} \\[0.2em]
  \>\cpuRegGPe{c}{counter, 4th arg.} \\[0.2em]
  \>\cpuRegGPe{d}{data, 3rd arg.} \\[0.5em]
  %
  \>\cpuRegSPe{si}{source index, 2nd arg.} \\[0.2em]
  \>\cpuRegSPe{di}{destination index, 1st arg.} \\[0.2em]
  \>\cpuRegSPe{bp}{base pointer, callee saved} \\[0.2em]
  \>\cpuRegSPe{sp}{stack pointer} \\[0.5em]
  %
  \>\cpuRegGPr {8}{5th arg.} \\[0.2em]
  \>\cpuRegGPr {9}{6th arg.} \\[0.2em]
  \>\cpuRegGPr{10}{caller saved} \\[0.2em]
  \>\cpuRegGPr{11}{caller saved} \\[0.5em]
  %
  \>\cpuRegGPr{12}{callee saved} \\[0.2em]
  \>\cpuRegGPr{13}{callee saved} \\[0.2em]
  \>\cpuRegGPr{14}{callee saved} \\[0.2em]
  \>\cpuRegGPr{15}{callee saved} \\[1em]
  %
  \>\color{gray}\cpuRegSPr{ip}{instruction pointer} \\[0.2em]
  \>\color{gray}\cpuRegSPr{flags}{condition flags}
\end{tabbing}
}% registers

В~64-разрядном режиме\footcite[3.4.1.1
General-Purpose Registers in~64-Bit Mode]{Intel-v1:2014:en}, размер операнда определяет количество действительных бит в~регистре"~приёмнике:
\begin{itemfeature}
  \item 64-битные операнды порождают 64-битный результат в~регистре"~приёмнике.
  \item 32-битные операнды порождают 32-битный результат, дополняемый нулями до~64-битного значения в регистре"~приёмнике.
  \item 8-~и 16-битные операнды порождают 8-~и 16-битные результаты. Старшие 56~или 48~бит, соответственно, в~регистре"~приёмнике остаются без~изменений. Если 8-~или 16-разрядная операция планируется для~вычисления 64-разрядного адреса, то необходимо в~явном виде дополнить знаковым битом до~полного 64-битного регистра.
\end{itemfeature}



%%=======================
\section{Формы операндов}
%%=======================
\begin{flushleft}
\newcommand*{\E}[1]{r\ensuremath{_{\text{#1}}}}%
\newcommand*{\by}{\ensuremath{\cdot}}%
%
\small\noindent\begin{tabular}{@{}l>{\ttfamily}l>{\ttfamily}ll@{}}
  \toprule
  Тип       & \textrm{Форма}     & \textrm{Значение операнда}     & Адресация \\
  \midrule
  константа & \$Imm              & Imm                            & непосредственная \\[0.5em]

  регистр   & \E{a}              & R[\E{a}]                       & регистровая \\[0.5em]

  память    & Imm                & M[Imm]                         & абсолютная \\
  память    &    (\E{a})         &     M[R[\E{a}]]                & косвенная \\
  память    & Imm(\E{b})         & M[Imm+R[\E{b}]]                & по~базе со~смещением \\
  память    & Imm(\E{b},\E{i})   & M[Imm+R[\E{b}]+R[\E{i}]]       & индексная \\
  память    & Imm(\E{b},\E{i},s) & M[Imm+R[\E{b}]+R[\E{i}]\by{s}] & индексная с~масштабированием \\
  \bottomrule
\end{tabular}\end{flushleft}

Положим, что в~перечисленных ниже адресах памяти и регистрах сохранены указанные значения. Заполните таблицу, вычислив значения операндов.
\begin{center}
  \ttfamily\small
  \begin{tabular}[t]{@{}ll@{}}
    \textrm{Адрес} & \textrm{Значение} \\
    \midrule
    0x100 & 0xFF \\
    0x104 & 0xAB \\
    0x108 & 0x13 \\
    0x10C & 0x11 \\[0.8em]

    \textrm{Регистр} & \textrm{Значение} \\
    \midrule
    \%rax & 0x100 \\
    \%rcx &   0x1 \\
    \%rdx &   0x3 \\
  \end{tabular}\hspace{3cm}
  %
  \newcommand*{\ans}[1]{\ansfw{1.5cm}{#1}}%
  %
  \begin{tabular}[t]{@{}ll@{}}
    \textrm{Операнд} & \textrm{Значение} \\
    \midrule
               \%rax & \ans{0x100} \\
               0x104 & \ans{0xAB} \\
             \$0x108 & \ans{0x108} \\
             (\%rax) & \ans{0xFF} \\
            4(\%rax) & \ans{0xAB} \\
      9(\%rax,\%rdx) & \ans{0x11} \\
    260(\%rcx,\%rdx) & \ans{0x13} \\
      0xFC(,\%rcx,4) & \ans{0xFF} \\
     (\%rax,\%rdx,4) & \ans{0x11} \\
  \end{tabular}
\end{center}



%%==========================
\section{Копирование данных}
%%==========================
{\small\ttfamily\begin{longtable}[l]{@{}lll>{\rmfamily}l@{}}
  \toprule
  \multicolumn{2}{@{}l}{\textrm{Инструкция}} & \textrm{Действие}  & Пояснение \\
  \endfirsthead
  \midrule
  \textbf{mov}   & S,D & D \(\leftarrow\) S & скопировать \\
          movb   &     &                    & \phantom{скопировать} байт \\
          movw   &     &                    & \phantom{скопировать} слово \\
          movl   &     &                    & \phantom{скопировать} двойное слово \\
          movq   &     &                    & \phantom{скопировать} четверное слово \\
  movabsq        & I,R & R \(\leftarrow\) I & \\[0.5em]

  \textbf{movz}  & S,R & R \(\leftarrow\) \textrm{ZeroExtend(\texttt{S})} & скопировать и расширить без~учёта знака \\
          movzbw &     &  & \\
          movzbl &     &  & \\
          movzwl &     &  & \\
          movzbq &     &  & \\
          movzwq &     &  & \\[0.5em]

  \textbf{movs}  & S,R & R \(\leftarrow\) \textrm{SignExtend(\texttt{S})} & скопировать и расширить с~учётом знака \\
          movsbw &     &  & \\
          movsbl &     &  & \\
          movswl &     &  & \\
          movsbq &     &  & \\
          movswq &     &  & \\
          movslq &     &  & \\
  cltq           &     & \%rax \(\leftarrow\) \textrm{SignExtend(\texttt{\%eax})} & \\
  \bottomrule
\end{longtable}}



%%=================
\paragraph{Пример.}
%%=================
Рассмотрим следующий код (\code{move.c}):

\cfile{projects/sem06/move.c}

\noindent Вызвав команду\footnote{Команда-синоним \texttt{asmlst} рассмотрена ранее в~разделе~\ref{alias:asmlst} на~странице~\pageref{alias:asmlst}.}:

\console/$ asmlst move.c/%$

\noindent получим ассемблерный листинг:

\precomment/xp: %rdi, y: %rsi/
\vspace{-1.7\baselineskip}
\gasfile[firstline=5, lastline=8]{projects/sem06/move.s}



%%=====================
\paragraph{Примечания:}
%%=====================
\phantom{some invisible text...}

\smallskip
\begin{enumerate}
\item Пользователи \name{Windows} могут отметить, что у~них в~коде аргументы передаются через регистры \code{\%rcx} и \code{\%rdx}. Это связано с~тем, что разные платформы могут использовать разные соглашения. Чтобы получить листинг в~соглашениях, принятых в~\name{Unix}, добавьте ключ \code{-mabi=sysv}:

\console/$ asmlst -mabi=sysv move.c/%$

\item В~\name{Linux} тип \code{long} означает длинное целое число (8~байт), а под~\name{Windows} это обычное целое число (4~байта). Используйте здесь и далее \code{long}~\code{long} или \code{int64\_t} из~библиотеки \code{<inttypes.h>}.
\end{enumerate}



%%===================================
\section{Работа с~программным стеком}
%%===================================
\begin{flushleft}\small\ttfamily\begin{tabular}[l]{@{}lll>{\rmfamily}l@{}}
  \toprule
  \multicolumn{2}{@{}l}{\textrm{Инструкция}} & \textrm{Действие}  & Пояснение \\
  \midrule
  pushq & S & R[\%rsp]    \(\leftarrow\) R[\%rsp] - 8; & поместить в~стек четверное слово \\
        &   & M[R[\%rsp]] \(\leftarrow\) S             & \\[0.5em]

  popq  & D & D        \(\leftarrow\) M[R[\%rsp]];  & взять из~стека четверное слово \\
        &   & R[\%rsp] \(\leftarrow\) R[\%rsp] + 8  & \\
  \bottomrule
\end{tabular}\end{flushleft}

Чтобы разместить в~стеке четверное слово, необходимо вначале уменьшить указатель стека на~8, а затем записать целевое значение по~новому адресу вершины стека. Таким образом, поведение инструкции \code{pushq~\%rbp} эквивалентно следующей паре инструкций:
\begin{gascode}
  subq $8, %rsp      # decrement stack pointer
  movq %rbp, (%rsp)  # store %rbp on stack
\end{gascode}

\noindent за~исключением того, что инструкция \code{pushq} представляется в~машинном коде всего лишь одним байтом, в~то время как пара инструкций, представленных выше, требует 8~байт.

Чтобы изъять четверное слово из~стека, необходимо вначале прочитать его с~вершины стека, а затем увеличить указатель стека на~8. Таким образом, инструкция \code{popq~\%rax} эквивалентна следующей паре инструкций:
\begin{gascode}
  movq (%rsp), %rax  # read %rax from stack
  addq $8, %rsp      # increment stack pointer
\end{gascode}



%%================
\WhatToReadSection
%%================
\citeauthor[глава~3, стр.~197--209]{Bryant:2022:ru}



%%===============
\ExercisesSection
%%===============
\begin{exercise}
\item Заполните пропущенные суффиксы команды \code{mov}, основываясь на~размере операндов:
{\newcommand*{\ans}{\ansvw}
\begin{gascode*}{texcl}
  mov  %eax, (%rsp)        # \ans{l}
  mov  (%rax), %dx         # \ans{w}
  mov  $0xFF, %bl          # \ans{b}
  mov  (%rsp,%rdx,4), %dl  # \ans{b}
  mov  (%rdx), %rax        # \ans{q}
  mov  %dx, (%rax)         # \ans{w}
\end{gascode*}
}


\item Каждая из~следующих строк вызывает ошибку ассемблера. Поясните, в~чём проблема.
\begin{gascode}
  movb $0xF, (%ebx)
  movl %rax, (%rsp)
  movw (%rax), 4(%rsp)
  movb %al, %sl
  movq %rax, $0x123
  movl %eax, %rdx
  movb %si, 8(%rbp)
\end{gascode}


\item Предположим, что переменные \code{sp} и \code{dp} объявлены как:
\begin{ccode*}{linenos=false}
src_t  *sp;
dest_t *dp;
\end{ccode*}
где \code{src\_t} и \code{dest\_t} некоторые типы данных, объявленные при~помощи \code{typedef}. Мы хотим использовать инструкции копирования, чтобы реализовать операцию:

\cc/*dp = (dest_t) *sp;/

Положим, что значения \code{sp} и \code{dp} размещены в~регистрах \code{\%rdi} и \code{\%rsi}, соответственно. Для каждой записи в~таблице выпишите две инструкции, которые реализуют указанную операцию присваивания. Первая инструкция должна читать данные из~памяти, выполнять требуемое преобразование и записывать соответствующую часть регистра \code{\%rax}. Вторая инструкция затем должна записать эту часть регистра обратно в~память. В~обоих случаях частями могут быть \code{\%rax}, \code{\%eax}, \code{\%ax}, \code{\%al}, и они могут отличаться друг от~друга.

Помните, что при~выполнении преобразования в~\lang{C}, которое включает и изменение размера, и изменение <<знаковости>>, операция должна сначала изменить размер.

\begin{flushleft}\noindent\ttfamily\small
\newcommand*{\w}{12em}%
\newcommand*{\ans}{\ansfw{\w}}%
\begin{tabular}{@{}lll@{}}
  src\_t & dest\_t & \textrm{Инструкция} \\
  \midrule
  long          & long          & \ansfw*{\w}{movq (\%rdi), \%rax} \\
                &               & \ansfw*{\w}{movq	\%rax, (\%rsi)} \\

  char          & int           & \ans{movsbl (\%rdi), \%eax} \\
                &               & \ans{movl   \%eax, (\%rsi)} \\

  char          & unsigned      & \ans{movsbl (\%rdi), \%eax} \\
                &               & \ans{movl   \%eax, (\%rsi)} \\

  unsigned char & long          & \ans{movzbq (\%rdi), \%rax} \\
                &               & \ans{movq   \%rax, (\%rsi)} \\

  int           & char          & \ans{movl (\%rdi), \%eax} \\
                &               & \ans{movb \%al, (\%rsi)} \\

  unsigned      & unsigned char & \ans{movl (\%rdi), \%eax} \\
                &               & \ans{movb \%al, (\%rsi)} \\

  char          & short         & \ans{movsbw (\%rdi), \%ax} \\
                &               & \ans{movw   \%ax, (\%rsi)} \\
\end{tabular}\end{flushleft}


\item Функция имеет прототип:

\cc/void decode (long *xp, long *yp, long *zp);/

и компилируется в~ассемблерный код, приведённый ниже:

\precomment/xp: %rdi, yp: %rsi, zp: %rdx/
\vspace{-0.7\baselineskip}
\gasfile[firstline=5, lastline=12]{projects/sem06/decode.s}

Параметры \code{xp}, \code{yp} и \code{zp} хранятся в~регистрах \code{\%rdi}, \code{\%rsi} и \code{\%rdx}, соответственно. Выпишите эквивалентный исходный~\lang{C} код для~функции \code{decode}.

\end{exercise}
